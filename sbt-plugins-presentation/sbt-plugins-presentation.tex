\documentclass{beamer}
\usepackage[utf8]{inputenc}
\usepackage{lmodern}
\usepackage{listings}
\setbeamerfont{structure}{family=\rmfamily,shape=\itshape} 

\usecolortheme{seahorse}
\usecolortheme{rose}
\setbeamercolor{title}{fg=red!80!black,bg=red!20!white}
\usetheme{Pittsburgh}

\title{SBT plugins}
\subtitle{an easy way to develop scala applications}
\author{José Miguel Martínez Carrasco \\ \texttt{jose.miguel@springer.com}}
\institute[Springer]{Springer}
\date{\today}

\begin{document}

%título
\begin{frame}[plain] 
  \titlepage
\end{frame}

%tabla de contenidos
\begin{frame}
  \frametitle{Outline}
  \tableofcontents
\end{frame}

%introducción
\begin{frame}{Introduction}

  \section{Motivation}

  Simple Extensibility.
  	
  Plugins provide a means of injecting and augmenting settings and commands. Since plugins are just Scala code, you can package and share plugins between projects.

\end{frame}

\begin{frame}{Create plugin}

  \begin{itemize}

    \item set up project

    \item implement plugin

  \end{itemize}
\end{frame}

\begin{frame}[fragile]
  \frametitle{Set up}

  \lstset{language=ksh,basicstyle=\ttfamily}
  \begin{lstlisting}[frame=none]
    mkdir my-sbt-plugin

    cd my-sbt-plugin

    touch build.sbt

    mkdir project

    touch project/build.properties

    touch project/plugins.sbt

    mkdir -p src/main/scala
  \end{lstlisting}
\end{frame}


\begin{frame}{fragile}
  \frametitle{build.properties}

    \lstset{language=ksh,basicstyle=\ttfamily}
    \begin{lstlisting}[frame=none]
      sbt.version=0.11.2
    \end{lstlisting}
\end{frame}


\begin{frame}{Useful plugins}

\end{frame}

\begin{frame}
\end{frame}

\end{document}
