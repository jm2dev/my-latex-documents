\documentclass[utf8]{beamer}

\usepackage{lmodern}
\usepackage{listings}

\title{SBT plugins}
\author{José Miguel Martínez Carrasco}
\institute[Springer]{Springer}
\date{\today}

\setbeamercolor{postit}{fg=black,bg=white!80!black}
\logo{\includegraphics[height=0.5cm]{springer_horse.png}}

% scala: http://tihlde.org/~eivindw/latex-listings-for-scala/
% "define" Scala
\lstdefinelanguage{scala}{
  morekeywords={abstract,case,catch,class,def,%
    do,else,extends,false,final,finally,%
    for,if,implicit,import,match,mixin,%
    new,null,object,override,package,%
    private,protected,requires,return,sealed,%
    super,this,throw,trait,true,try,%
    type,val,var,while,with,yield},
  otherkeywords={=>,<-,<\%,<:,>:,\#,@, <+=},
  sensitive=true,
  morecomment=[l]{//},
  morecomment=[n]{/*}{*/},
  morestring=[b]",
  morestring=[b]',
  morestring=[b]"""
}

\usepackage{color}
\definecolor{dkgreen}{rgb}{0,0.6,0}
\definecolor{gray}{rgb}{0.5,0.5,0.5}
\definecolor{mauve}{rgb}{0.58,0,0.82}
 
% Default settings for code listings
\lstset{frame=tb,
  language=scala,
  aboveskip=3mm,
  belowskip=3mm,
  showstringspaces=false,
  columns=flexible,
  basicstyle={\small\ttfamily},
  numbers=left,
  numberstyle=\tiny\color{gray},
  keywordstyle=\color{blue},
  commentstyle=\color{dkgreen},
  stringstyle=\color{mauve},
  frame=single,
  breaklines=true,
  breakatwhitespace=true
  tabsize=3
}

\begin{document}
\frame{\maketitle}

\section{Introduction}
\frame{\tableofcontents[currentsection]}

\subsection{Motivation}
\frame{
  Simple Extensibility.

  \begin{beamercolorbox}[shadow=true, rounded=true]{postit}	  
    Plugins provide a means of injecting and augmenting settings and commands. Since plugins are just Scala code, you can package and share plugins between projects.  
  \end{beamercolorbox}
}

\section{Create a plugin}
\frame{\tableofcontents[currentsection]}

\frame{
  \begin{itemize}
    \item{Create \emph{your project} folder}
    \item{build.sbt}
    \item{project/build.properties}
    \item{project/plugins.sbt}
    \item{Scala class extending Plugin class.}
  \end{itemize}
}

\subsection{build.sbt}
\begin{frame}[fragile]

  \begin{lstlisting}[title={build.sbt}]
    sbtPlugin := true

    name := "sbt-xslt-plugin"

    organization := "com.jm2dev"

    version := "0.1.6"

    scalaVersion := "2.9.1"

    seq(scriptedSettings: _*)

    libraryDependencies ++= Seq(
      "org.scalatest" %% "scalatest" % "1.6.1" % "test",
      "net.sf.saxon" % "Saxon-HE" % "9.4"
    )    
  \end{lstlisting}

\end{frame}

\subsection{build.properties}
\begin{frame}[fragile]

  \begin{lstlisting}[title={project/build.properties}]
    sbt.version=0.11.2
  \end{lstlisting}

\end{frame}

\subsection{plugins.sbt}
\begin{frame}[fragile]
  \begin{lstlisting}[title={project/plugins.sbt}]
    
    libraryDependencies <+= (sbtVersion) { (version) =>
  "org.scala-tools.sbt" %% "scripted-plugin" % version
}

resolvers ++= Seq (
   "coda" at "http://repo.codahale.com"
)
  \end{lstlisting}

\end{frame}

\subsection{MyPlugin.scala}
\begin{frame}[fragile]
  \begin{lstlisting}[title={src/main/scala/MyPlugin.scala}]

    import sbt._
    import Keys._

    object MyPlugin extends Plugin
    {
      override lazy val settings = Seq(commands += myCommand)

      lazy val myCommand = 
        Command.command("hello") { (state: State) =>
          println("Hi!")
          state
        }
    }
  \end{lstlisting}

\end{frame}

\section{Plugins I like}
\frame{\tableofcontents[currentsection]}

\subsection{idea}
\frame{
  \frametitle{idea}
  
  \begin{beamercolorbox}[shadow=true, rounded=true]{postit}
    Generates Intellij Idea project configuration.
  \end{beamercolorbox}
}

\begin{frame}[fragile]

  \begin{lstlisting}[title={plugins.sbt}]
    resolvers += "sbt-idea-repo" at "http://mpeltonen.github.com/maven/"

    addSbtPlugin("com.github.mpeltonen" % "sbt-idea" % "1.0.0")
  \end{lstlisting}

  \begin{lstlisting}[title={usage}]
    gen-idea
    
    gen-idea no-classifiers
  \end{lstlisting}

\end{frame}

\subsection{git}
\frame{
  \frametitle{git}
  
  \begin{beamercolorbox}[shadow=true, rounded=true]{postit}
    codigo de git
  \end{beamercolorbox}
}

\subsection{assembly}
\frame{
  \frametitle{assembly}
  
  \begin{beamercolorbox}[shadow=true, rounded=true]{postit}
    codigo de assembly
  \end{beamercolorbox}
}

\subsection{web}
\frame{
  \frametitle{web}
  
  \begin{beamercolorbox}[shadow=true, rounded=true]{postit}
    codigo de web
  \end{beamercolorbox}
}

\subsection{release}
\frame{
  \frametitle{release}
  
  \begin{beamercolorbox}[shadow=true, rounded=true]{postit}
    codigo de release
  \end{beamercolorbox}
}

\subsection{implicitly}
\frame{
  \frametitle{implicitly}
  
  \begin{beamercolorbox}[shadow=true, rounded=true]{postit}
    codigo de implicitly
  \end{beamercolorbox}
}

\section{Conclusions}
\frame{
  \begin{beamercolorbox}[shadow=true, rounded=true]{postit}	  
    Reusability.  
  \end{beamercolorbox}

  \begin{beamercolorbox}[shadow=true, rounded=true]{postit}	  
    Convention over configuration.  
  \end{beamercolorbox}
  
  \begin{beamercolorbox}[shadow=true, rounded=true]{postit}	  
    Simplicity.  
  \end{beamercolorbox}

}

\section*{Outline}
\frame{\tableofcontents}

\end{document}