% Created 2013-08-18 Sun 20:39
\documentclass[bigger]{beamer}
\usepackage[utf8]{inputenc}
\usepackage[T1]{fontenc}
\usepackage{fixltx2e}
\usepackage{graphicx}
\usepackage{longtable}
\usepackage{float}
\usepackage{wrapfig}
\usepackage{soul}
\usepackage{textcomp}
\usepackage{marvosym}
\usepackage{wasysym}
\usepackage{latexsym}
\usepackage{amssymb}
\usepackage{hyperref}
\tolerance=1000
\mode<beamer>{\usetheme{Pittsburgh}\setbeamercolor{postit}{fg=black,bg=white!80!black}\usecolortheme{fly}}
\providecommand{\alert}[1]{\textbf{#1}}

\title{PaaS :: OpenShift}
\author{Jose Miguel Martinez Carrasco, Héctor Rivas}
\date{2013-08-11 Sun}
\hypersetup{
  pdfkeywords={openshift, PaaS, Agile},
  pdfsubject={},
  pdfcreator={Emacs Org-mode version 7.9.3f}}

\begin{document}

\maketitle

\begin{frame}
\frametitle{Outline}
\setcounter{tocdepth}{3}
\tableofcontents
\end{frame}

\section{Context}
\label{sec-1}
\begin{frame}
\frametitle{SaaS}
\label{sec-1-1}


is a software delivery model in which software and associated data are centrally hosted on the cloud. 

Oriented to the end user, he won't have to care about the software and the data, and will have universal access.
\begin{itemize}

\item Examples\\
\label{sec-1-1-1}%
Gmail, Github, prezi, SpringerLink

\end{itemize} % ends low level
\end{frame}
\begin{frame}
\frametitle{IaaS}
\label{sec-1-2}


offers virtual resources for machines, disk, network, load balancers, ussually on-demand from pools in data centers.

Oriented to ops, that won't care about physical maintenance, can provision on-demand, etc.
\begin{itemize}

\item Examples\\
\label{sec-1-2-1}%
Amazon, rackspace, OpenStack

\end{itemize} % ends low level
\end{frame}
\begin{frame}
\frametitle{PaaS}
\label{sec-1-3}


provides a computing platform and a solution stack as a service (java, php, etc), taking care of autodeployment, scaling, etc.

Oriented to the developers, they won't care how the underlying servers and systems (meanwhile they follow some rules).
\begin{itemize}

\item Examples\\
\label{sec-1-3-1}%
Heroku, Google App Engine, Openshift

\end{itemize} % ends low level
\end{frame}
\begin{frame}
\frametitle{What can PaaS offer?}
\label{sec-1-4}


\begin{itemize}
\item deploy applications as easy as pushing code.
\item developers will focus on the application instead of on the underlying platform.
\item create new environments as needed, without any extra overhead.
\item scale on demand and get the best from our resources.
\item have isolated and homogenous platform
\end{itemize}
\end{frame}
\begin{frame}
\frametitle{What are the benefits?}
\label{sec-1-5}
\begin{itemize}

\item People working on the right tasks\\
\label{sec-1-5-1}%
\begin{itemize}
\item developers on delivering end user features
\item systems engineers on delivering a platform
\end{itemize}


\item Flexibility
\label{sec-1-5-2}%
\begin{itemize}
\item easy environments creation
\item fast services deployment
\item maintenance overhead reduction
\item tasks simplification
\end{itemize}

\end{itemize} % ends low level
\end{frame}
\begin{frame}
\frametitle{I buy it, I want it now!!!}
\label{sec-1-6}


OK, but this does not come for free.

\begin{itemize}
\item we need a PaaS solution
\item we need to learn how to use it
\item we need to adapt our software
\item we need to change our delivery process
\end{itemize}
\end{frame}
\begin{frame}
\frametitle{Use a commercial hosted solution}
\label{sec-1-7}
\begin{itemize}

\item Examples\\
\label{sec-1-7-1}%
Google App Engine, Heroku, Open Shift online, etc


\item Disadvantages\\
\label{sec-1-7-2}%
\begin{itemize}
\item expensive
\item in their datacenters
\item they decide the technology stack
\item sometimes closed solutions
\item their own stric policies and rules
\end{itemize}

\end{itemize} % ends low level
\end{frame}
\begin{frame}
\frametitle{Use your own PaaS solution}
\label{sec-1-8}
\begin{itemize}

\item Examples\\
\label{sec-1-8-1}%
Openshift, CloudFoundry, Flynn, etc or your own implementation.

\begin{itemize}
\item You need to setup, install and maintain the platform
\item You need to custumize it if need
\end{itemize}
  
\end{itemize} % ends low level
\end{frame}
\begin{frame}
\frametitle{Learning curve and adapt your software}
\label{sec-1-9}


OK, soon you will get the benefits, but you 

\begin{itemize}
\item have to learn how to use the platforms, the tools, etc.
\item adapt the applications to a set of conventions (configuration, path, etc)
\end{itemize}
\end{frame}
\begin{frame}
\frametitle{Casper: Where are we now?}
\label{sec-1-10}


to setup a new service, or deploy a new environment: 

\begin{itemize}
\item request servers
\item provision and set up them
\item configure applications
\item set up a pipeline
\item set up load balancing
\item request for DNS entries
\item map the application  to servers
\item etc
\end{itemize}
\end{frame}
\section{Solution: OpenShift}
\label{sec-2}
\begin{frame}
\frametitle{What is it?}
\label{sec-2-1}


OpenShift is a cloud computing platform as a service product from Red Hat. A version for private cloud is named OpenShift Enterprise.

The software that runs the service is open-sourced under the name
OpenShift Origin, and is available on GitHub. 

Developers can use Git to deploy web applications in different languages on the platform.

Uniquely, OpenShift also supports binary programs that are web
applications, so long as they can run on Red Hat Enterprise Linux. 
This allows the use of arbitrary languages and frameworks. OpenShift takes care of maintaining the services underlying the application and scaling the application as needed.
\end{frame}
\begin{frame}[fragile]
\frametitle{Simplify the lifecycle}
\label{sec-2-2}


Getting a Java app with a MySQL backend deployed onto OpenShift is as
easy as executing two commands:


\begin{verbatim}
rhc app create MyApp jbossews
rhc cartridge add mysql-5.1 -a MyApp
\end{verbatim}

These two commands create your ``server'' and install and configure Tomcat (via JBoss EWS), MySQL, a git repository on the server, and a simple web application. You can now visit your application on the web at:

\href{http://MyApp-MyDomain.rhcloud.com/}{http://MyApp-MyDomain.rhcloud.com/}
\end{frame}
\begin{frame}
\frametitle{Simplify the lifecycle (II)}
\label{sec-2-3}


Now we do not need to worry about servers, DNS entries, load
balancing, high availability, etc.

Everything is provided by the platform infrastructure, that is also
the reason Openshift is PaaS (Platform as a service).

It supports the most popular technologies like Java, ruby, nodejs,
etc; and allow us to extend it via cartridges to support additional applications.
\end{frame}
\begin{frame}
\frametitle{What does it provide?}
\label{sec-2-4}

 
\begin{itemize}
\item Rapid deployment
\item Early feedback
\item Focus on the app not the infrastructure
\item Autoscaling
\item DIY: cartridges
\end{itemize}
\end{frame}
\section{References}
\label{sec-3}
\begin{frame}
\frametitle{Links}
\label{sec-3-1}


\begin{itemize}
\item \href{http://www.slideshare.net/fallenpegasus/openshift-openstack-fedora-awesome}{Openshift and openstack}.
\item \hyperref[ttp-www.slideshare.net-jdewinne-cloud-development-using-play-scala-and-openshift]{Play Scala and Openshift}.
\end{itemize}
\end{frame}

\end{document}
